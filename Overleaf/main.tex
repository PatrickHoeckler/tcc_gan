\documentclass[
	% -- opções da classe memoir --
	12pt,				% tamanho da fonte
	%openright,			% capítulos começam em pág ímpar (insere página vazia caso preciso)
	oneside,			% para impressão no anverso. Oposto a twoside
	a4paper,			% tamanho do papel. 
	% -- opções da classe abntex2 --
	chapter=TITLE,		% títulos de capítulos convertidos em letras maiúsculas
	section=TITLE,		% títulos de seções convertidos em letras maiúsculas
	%subsection=TITLE,	% títulos de subseções convertidos em letras maiúsculas
	%subsubsection=TITLE,% títulos de subsubseções convertidos em letras maiúsculas
	% -- opções do pacote babel --
	english,
	brazil
	]{abntex2}

\usepackage{setup/ufscthesisA4-alf}
\usepackage{setup/my_styles}

% ---
% Filtering and Mapping Bibliographies
% ---
% Pacotes de citações
% ---
\usepackage{csquotes}
\usepackage[backend = biber, style = abnt]{biblatex}
% FIXME Se desejar estilo numérico de citações,  comente a linha acima e descomente a linha a seguir.
% \usepackage[backend = biber, style = numeric-comp]{biblatex}

\setlength\bibitemsep{\baselineskip}
\DeclareFieldFormat{url}{Disponível~em:\addspace\url{#1}}
\NewBibliographyString{sineloco}
\NewBibliographyString{sinenomine}
\DefineBibliographyStrings{brazil}{%
	sineloco     = {\mkbibemph{S\adddot l\adddot}},
	sinenomine   = {\mkbibemph{s\adddot n\adddot}},
	andothers    = {\mkbibemph{et\addabbrvspace al\adddot}},
	in			 = {\mkbibemph{In:}}
}

\addbibresource{references.bib} % Seus arquivos de referências

% ---
\DeclareSourcemap{
	\maps[datatype=bibtex]{
		% remove fields that are always useless
		\map{
			\step[fieldset=abstract, null]
			\step[fieldset=pagetotal, null]
		}
		% remove URLs for types that are primarily printed
%		\map{
%			\pernottype{software}
%			\pernottype{online}
%			\pernottype{report}
%			\pernottype{techreport}
%			\pernottype{standard}
%			\pernottype{manual}
%			\pernottype{misc}
%			\step[fieldset=url, null]
%			\step[fieldset=urldate, null]
%		}
		\map{
			\pertype{inproceedings}
			% remove mostly redundant conference information
			\step[fieldset=venue, null]
			\step[fieldset=eventdate, null]
			\step[fieldset=eventtitle, null]
			% do not show ISBN for proceedings
			\step[fieldset=isbn, null]
			% Citavi bug
			\step[fieldset=volume, null]
		}
	}
}

% ---
% Informações de dados para CAPA e FOLHA DE ROSTO
% ---
\autor{Patrick Hoeckler}
\titulo{An Analysis of Techniques for Building Generative Adversarial Networks}
%\subtitulo{Subtítulo (se houver)}
\orientador{Prof. Fabrício de Oliveira Ourique, Dr.}
% Coordenador do curso.
\coordenador{Prof. Fabrício de Oliveira Ourique, Dr.}
% Ano em que o trabalho foi defendido.
\ano{2021}
% Data em que ocorreu a defesa.
\data{05 de Maio de 2021}
% Cidade em que ocorreu a defesa.
\local{Araranguá}
\instituicaosigla{UFSC}
\instituicao{Universidade Federal de Santa Catarina}
\tipotrabalho{Trabalho de Conclusão de Curso}
\formacao{Bacharel em Engenharia de Computação}
\nivel{Bacharel}
\programa{Curso de Graduação em Engenharia de Computação}
\centro{Centro de Ciências, Tecnologias e Saúde do Campus Araranguá}

\preambulo {
    \imprimirtipotrabalho~do~\imprimirprograma~do~\imprimircentro~da~\imprimirinstituicao~para~a~obtenção~do~título~de~\imprimirformacao.
}

% ---
% Configurações de aparência do PDF final
% ---
% alterando o aspecto da cor azul
\definecolor{blue}{RGB}{41,5,195}
% informações do PDF
\makeatletter
\hypersetup{
     	%pagebackref=true,
		pdftitle={\@title}, 
		pdfauthor={\@author},
    	pdfsubject={\imprimirpreambulo},
	    pdfcreator={LaTeX with abnTeX2},
		pdfkeywords={ufsc, latex, abntex2}, 
		colorlinks=true,       		% false: boxed links; true: colored links
    	linkcolor=black,%blue,          	% color of internal links
    	citecolor=black,%blue,        		% color of links to bibliography
    	filecolor=black,%magenta,      		% color of file links
		urlcolor=black,%blue,
		bookmarksdepth=4
}
\makeatother
% ---

\siglalista{MNIST}{Modified National Institute of Standards and Technology database}
\siglalista{CIFAR}{Canadian Institute for Advanced Research}

\siglalista{KNN}{K-Nearest Neighbours}
\siglalista{SVM}{Support Vector Machine}
\siglalista{CNN}{Convolutional Neural Network}
\siglalista{FC}{Fully Connected}

\siglalista{DL}{Deep Learning}
\siglalista{MLP}{Multilayer Perceptron}
\siglalista{LSTM}{Long Short-Term Memory}
\siglalista{RNN}{Recurrent Neural Network}
\siglalista{ResNet}{Residual Networks}
\siglalista{MSE}{Mean Squared Error}

\siglalista{GAN}{Generative Adversarial Network}
\siglalista{DCGAN}{Deep Convolutional GAN}
\siglalista{CGAN}{Conditional GAN}
\siglalista{WGAN}{Wassertein GAN}
\siglalista{WGAN-GP}{WGAN with Gradient Penalization}

\siglalista{EM}{Earth-Mover}

\siglalista{ReLU}{Rectified Linear Unit}
\siglalista{tanh}{hyperbolic tangent}

\siglalista{SGD}{Stochastic Gradient Descent}
\siglalista{Adam}{Adaptive Moment Estimation}

\siglalista{FID}{Fréchet Inception Distance}
\siglalista{FCD}{Fréchet Classifier Distance}
\siglalista{IS}{Inception Score}
\siglalista{CS}{Classifier Score}

\siglalista{BS}{Batch Size}
\siglalista{BN}{Batch Normalization}
\siglalista{UP}{Upscaling Method}
\siglalista{OPT}{Optimizer}
\siglalista{LDIM}{Dimensions of latent space}
\siglalista{CLIP}{Clipping value for the parameters of the critic}
\siglalista{NCRIT}{Number of updates to critic before update to generator}
\siglalista{SMOOTH}{Value of one-sided label smoothing}
\simbololista{sigmoid}{\ensuremath{\sigma}}{Sigmoid function}
\simbololista{theta}{\ensuremath{\theta}}{Trainable Parameters of the Neural Network}
\simbololista{lambda}{\ensuremath{\lambda}}{Regularization strength}
\simbololista{epsilon}{\ensuremath{\epsilon}}{Computational stability term}
\simbololista{latent_space}{\ensuremath{\mathcal{Z}}}{Latent space of the generator}
\simbololista{input_space}{\ensuremath{\mathcal{X}}}{Input space}
\simbololista{expected_value}{\ensuremath{\mathbb{E}}}{Expected value}

\simbololista{learning_rate}{\ensuremath{\eta}}{Learning Rate}
\simbololista{beta_1}{\ensuremath{\beta_1}}{Momentum hyperparameter of Adam}

% compila a lista de abreviaturas e siglas e a lista de símbolos
\makenoidxglossaries

% compila o indice
\makeindex

\begin{document}

\selectlanguage{english}  % Seleciona o idioma do documento (conforme pacotes do babel)
\frenchspacing  % Retira espaço extra obsoleto entre as frases.
\OnehalfSpacing  % Espaçamento 1.5 entre linhas

% Corrige justificação
%\sloppy

% ELEMENTOS PRÉ-TEXTUAIS
% ---
% Capa
% ---
\imprimircapa
% ---

% ---
% Folha de rosto
% (o * indica que haverá a ficha bibliográfica)
% ---
\imprimirfolhaderosto*
% ---

% ---
% Inserir a ficha bibliografica
% ---
% http://ficha.bu.ufsc.br/
\begin{fichacatalografica}
	\includepdf{beforetext/Ficha_Catalografica.pdf}
\end{fichacatalografica}
% ---

% ---
% Inserir folha de aprovação
% ---
\begin{folhadeaprovacao}
	\OnehalfSpacing
	\centering
	\imprimirautor\\%
	\vspace*{10pt}		
	\textbf{\imprimirtitulo}%
	\ifnotempty{\imprimirsubtitulo}{:~\imprimirsubtitulo}\\%
	%		\vspace*{31.5pt}%3\baselineskip
	\vspace*{\baselineskip}
	%\begin{minipage}{\textwidth}
	% ~do~\imprimirprograma~do~\imprimircentro~da~\imprimirinstituicao~para~a~obtenção~do~título~de~\imprimirformacao.
	Este~\imprimirtipotrabalho~foi julgado adequado para obtenção do Título de \imprimirformacao ~e aprovado em sua forma final pelo~\imprimirprograma. \\
		\vspace*{\baselineskip}
	\imprimirlocal, \imprimirdata. \\
	\vspace*{2\baselineskip}
	\assinatura{\OnehalfSpacing\imprimircoordenador \\ \imprimircoordenadorRotulo~do Curso}
	\vspace*{2\baselineskip}
	\textbf{Banca Examinadora:} \\
	\vspace*{\baselineskip}
	\assinatura{\OnehalfSpacing\imprimirorientador \\ Orientador}
	%\end{minipage}%
	\vspace*{\baselineskip}
	\assinatura {
	    Prof. Antônio Carlos Sobieranski, Dr.\\
	    Avaliador \\
	    Universidade Federal de Santa Catarina
    }

	\vspace*{\baselineskip}
	\assinatura {
    	Prof. Alexandre Leopoldo Gonçalves, Dr.\\
    	Avaliador \\
    	Universidade Federal de Santa Catarina
	}


\end{folhadeaprovacao}
% ---

\begin{dedicatoria}
	\vspace*{\fill}
	\noindent
	\begin{adjustwidth*}{}{6.5cm}     
		For CEP
	\end{adjustwidth*}
\end{dedicatoria}
\begin{agradecimentos}
Eu gostaria muito que essa parte estivesse em branco, que eu pudesse dizer que fiz tudo sozinho e não tive nenhuma ajuda, mas não foi bem assim que aconteceu, eu não teria chegado até aqui se eu tivesse vindo sozinho. Tem algumas pessoas que eu gostaria de agradecer.

Mantendo o clichê, eu primeiro tenho que agradecer a minha mãe. Ela não teve muita influência direta nessa minha formação, eu até estava me questionando se fazia sentido incluir ela aqui já que não parecia que ela teve muito a ver com essa parte da minha vida. Mas aí eu percebi que eu estava sendo maluco, eu não consigo pensar em nenhuma pessoa que tenha feito mais por mim do que minha mãe, se olhar para as influências indiretas vai dar de perceber claramente que elas são bem maiores do que qualquer influência que os outros possam ter tido.

Eu tenho quase certeza de que se tirassem qualquer outra pessoa da minha vida eu ainda teria conseguido chegar aqui, seria bem mais sofrido, mas eu teria chegado. Mas se tirasse a minha mãe eu nem sei se eu teria conseguido começar a faculdade. O clichê é clichê por um motivo, obrigado mãe.

Mas partindo pra influências mais diretas, meu pai, fazer a faculdade sem a ajuda que meu pai me deu teria sido muito mais sofrido. Ele foi quem me manteve alimentado e com um teto em cima da cabeça, eu não sei como eu teria feito sem tua ajuda pai.

Seguindo o baile para os meus colegas de apartamento, Gustavo (o Gino) e Caio. O Gino foi quem me apresentou esse curso e dividiu o apartamento comigo durante todo o curso, o Caio entro lá pelo meio do tempo pra trazer mais alegria e principalmente pra baratear mais o aluguel, sempre é bom. Bons tempos, boas lembraças, algumas não tão boas, mas ainda estamos no lucro.

A meus bons amigos que fiz no curso, em especial o Ramom, o Ricardo e o Felipe, grandes camaradas. Também tive o prazer de trabalhar com alguns professores aqui da UFSC que me fizeram aprender muita coisa e conseguir uma renda extra (sempre é bom), em especial gostaria de agradecer ao professor Marcelo Zannin da Rosa, ótima pessoa e com um ótimo gosto para camisetas. Espero que a bondade que vocês mostraram para mim possa voltar para vocês.

Também gostaria de agradecer meus avaliadores Alexandre L. Gonçalves e Antônio C. Sobieranski junto com meu orientador Fabrício de Oliveira Ourique, minhas aulas com vocês estão entre as melhores e as quais eu aprendi mais. O Fabrício também foi um ótimo orientador, além de me ajudar com o TCC me ajudou com várias dúvidas que eu tinha sobre o estágio.

Por último eu gostaria de agradecer a todos aqueles que estão lutando pela educação gratuita, com certeza essas pessoas tiveram um enorme impacto em tudo o que eu aprendi aqui na faculdade. Tudo o que eu fiz nesse TCC eu aprendi gratuitamente, eu acho incrível que eu possa sentar na minha casa e assistir cursos completos do MIT, de Harvard, de Stanford e todos os outros. Eu não teria aprendido tanto se não fosse por todas essas pessoas e eu torço que isso consiga chegar pra todo mundo um dia.

\end{agradecimentos}
% \include{beforetext/epigrafe}
\setlength{\absparsep}{18pt} % ajusta o espaçamento dos parágrafos do resumo

\begin{resumo}
	\SingleSpacing
	Generative Adversarial Networks (GANs) are a subcategory of Artificial Neural Networks where the objective is the generation of new data, they do that by modeling the probability distribution of real data, usually coming from a dataset, and sampling from the modeled distribution in order to produce original data that is similar, and optimally indistinguishable, from what was used in training.
	The principle behind GANs is based on a competition between two different networks, a discriminator who tries to distinguish real from fake data, and a generator who tries to fool the discriminator by producing data that is as close to the real one as possible.
	However, the competition between the networks makes training GANs be something notoriously difficult, instability and non-convergence are a common occurrence and many techniques have been proposed to improve not only the learning process, but also the quality of the generated results.
	The goal for this document was to analyse a number of the most common approaches and make an empirical evaluation of those, trying to apply the techniques in different datasets and seeing which configuration produces the best results. In the end there should be a roadmap that can be used to help guide the initial decisions about what method to use when constructing GANs for new and unknown situations.
	
	\textbf{Keywords}: Deep Learning. Neural Networks. Generative models. Generative Adversarial Networks. GAN.
\end{resumo}



\begin{resumo}[Resumo]
	\SingleSpacing
	\begin{otherlanguage*}{brazil}
		Generative Adversarial Networks (GANs) são uma subcategoria de Rede Neurais Artificiais onde o objetivo é a geração de novos dados, elas fazem isso tentando modelar a distribuição de probabilidades de dados reais, geralmente vindos de um dataset, e amostrando da distribuição modelada de modo a produzir dados originais que são similares, e idealmente indistinguíveis do que foi usado durante o treino.
    	O princípio por trás de GANs é baseado em uma competição entre duas redes distintas, um discriminador que tenta distinguir entre dados reais e falsos, e um gerador que tenta enganar o discriminador produzindo dados que são o mais perto possível dos dados reais.
    	Entretanto, a competição entre as duas redes faz do treinamento de GANs algo que é notoriamente difícil, instabilidade e não-convergência são ocorrências comuns e muitas técnicas foram propostas para melhorar não apenas o processo de aprendizado, mas também a qualidade dos resultados gerados.
    	O objetivo deste documento foi de analisar um número de abordagens mais comuns e realizar uma avaliação empírica destas, tentando aplicar as técnicas em diferentes datasets e observando qual configuração produz os melhores resultados. Ao fim deve haver um roteiro que pode ser usado para ajudar a guiar as decisões iniciais sobre qual método utilizar ao construir GANs para novas situações desconhecidas.
    	
    	\textbf{Palavras-chave}: Deep Learning. Neural Networks. Modelos generativos. Generative Adversarial Networks. GAN.
	\end{otherlanguage*}
\end{resumo}


{ %hidelinks
	\hypersetup{hidelinks}
	% inserir lista de ilustrações
	\pdfbookmark[0]{\listfigurename}{lof}
	\listoffigures*
	\cleardoublepage
	
	% inserir lista de quadros
    % 	\pdfbookmark[0]{\listofquadrosname}{loq}
    % 	\listofquadros*
    % 	\cleardoublepage
	
	% inserir lista de tabelas
% 	\pdfbookmark[0]{\listtablename}{lot}
% 	\listoftables*
% 	\cleardoublepage
	
	% inserir lista de siglas e lista de simbolos
	\imprimirlistadesiglas
	\imprimirlistadesimbolos
	
	% inserir o sumario
	\pdfbookmark[0]{\contentsname}{toc}
	\tableofcontents*
	\cleardoublepage
} %hidelinks


% -- ELEMENTOS TEXTUAIS ------------------------------------
\textual
\section{CGAN}
To implement a \gls{CGAN} it is only necessary to modify an existing \gls{GAN} to receive a conditional label. For these experiments, the same base used for the \gls{DCGAN} was used to construct the \gls{CGAN}, the only difference was the addition of another input for the label that will pass through an embedding layer and be incorporated into a channel of the other input, as discussed in \autoref{sub:cgan}.

\subsection{MNIST}
Training the \gls{CGAN} on the \gls{MNIST} dataset produced the results seen in \autoref{fig:cgan_mnist_metrics}.\begin{figure}[hbt]
    \centering
    \caption{Metrics when training a CGAN on MNIST}
    \includegraphics[width=\textwidth]{chapters/Experiments/CGAN/mnist_metrics.pdf}
    \fonte{From the author (2021)}
    \label{fig:cgan_mnist_metrics}
\end{figure}

These results show not only that the metrics have improved, but also that the training is much more stable, for all tests the training produced similar good results.

One of the main benefits of the \gls{CGAN} architecture is the fact that the resulting sample can be controlled by feeding the desired label to the generator. This can be seen in the samples produced by the best model (\texttt{BS=16 SMOOTH=0.9}) shown in \autoref{fig:cgan_mnist_samples}, where each row was conditioned to produce a different set of digits.
\begin{figure}[hbt]
    \centering
    \caption{Samples when training a CGAN on MNIST}
    \includegraphics[width=0.5\textwidth]{chapters/Experiments/CGAN/mnist_samples.png}
    \fonte{From the author (2021)}
    \label{fig:cgan_mnist_samples}
\end{figure}


\subsection{Fashion MNIST}
Training the \gls{CGAN} on the Fashion MNIST dataset produced the results seen in \autoref{fig:cgan_fashion_metrics}.
\begin{figure}[hbt]
    \centering
    \caption{Metrics when training a CGAN on Fashion MNIST}
    \includegraphics[width=\textwidth]{chapters/Experiments/CGAN/fashion_metrics.pdf}
    \fonte{From the author (2021)}
    \label{fig:cgan_fashion_metrics}
\end{figure}

The same behaviour seen for \gls{MNIST} can also be seen here, the overall quality increased, this time however the level of stability is not as strong. The samples produced by the best test (\texttt{BS=32}) are shown in \autoref{fig:cgan_fashion_samples}.
\begin{figure}[hbt]
    \centering
    \caption{Samples when training a CGAN on Fashion MNIST}
    \includegraphics[width=0.5\textwidth]{chapters/Experiments/CGAN/fashion_samples.png}
    \fonte{From the author (2021)}
    \label{fig:cgan_fashion_samples}
\end{figure}


\subsection{CIFAR-10}
Training the \gls{CGAN} on the \gls{CIFAR}-10 dataset produced the results seen in \autoref{fig:cgan_cifar_metrics}. In these experiments it was also tested the \gls{beta_1} term of Adam to see if some momentum in the updates would be beneficial.
\begin{figure}[hbt]
    \centering
    \caption{Metrics when training a CGAN on CIFAR-10}
    \includegraphics[width=\textwidth]{chapters/Experiments/CGAN/cifar_metrics.pdf}
    \fonte{From the author (2021)}
    \label{fig:cgan_cifar_metrics}
\end{figure}

To better understand how each component affects the results it is useful to highlight them separately. \autoref{fig:cgan_cifar_upsampling} shows how the metrics evolve for the different upsampling techniques. The same behaviour observed previously is also seen here, still the transposed convolutions perform better, followed by nearest neighbour and bilinear upsampling.
\begin{figure}[hbt]
    \centering
    \caption{Effects of upsampling when training a CGAN on CIFAR-10}
    \includegraphics[width=\textwidth]{chapters/Experiments/CGAN/cifar_upsampling.pdf}
    \fonte{From the author (2021)}
    \label{fig:cgan_cifar_upsampling}
\end{figure}

In \autoref{fig:cgan_cifar_beta1} it is possible to see the effect of the momentum term \gls{beta_1} in training. The results support the idea that momentum does not help for \gls{CIFAR}-10. Also note how the bad performing cases of transposed convolutions seen in \autoref{fig:cgan_cifar_upsampling} are because the use of momentum.
\begin{figure}[hbt]
    \centering
    \caption{Effects of momentum when training a CGAN on CIFAR-10}
    \includegraphics[width=\textwidth]{chapters/Experiments/CGAN/cifar_beta1.pdf}
    \fonte{From the author (2021)}
    \label{fig:cgan_cifar_beta1}
\end{figure}

Lastly, the effects of label smoothing can be seen in \autoref{fig:cgan_cifar_smoothing}. The impact is still not very significative, although it can be seen a small tendency for better values of the \gls{FCD}.
\begin{figure}[hbt]
    \centering
    \caption{Effects of label smoothing when training a CGAN on CIFAR-10}
    \includegraphics[width=\textwidth]{chapters/Experiments/CGAN/cifar_beta1.pdf}
    \fonte{From the author (2021)}
    \label{fig:cgan_cifar_smoothing}
\end{figure}

The samples produced by the best test (\texttt{$\beta_1$=0.0 UP=TrpConv SMOOTH=0.9}) are shown in \autoref{fig:cgan_cifar_samples}.
\begin{figure}[hbt]
    \centering
    \caption{Samples when training a CGAN on CIFAR-10}
    \includegraphics[width=0.65\textwidth]{chapters/Experiments/CGAN/cifar_samples.pdf}
    \fonte{From the author (2021)}
    \label{fig:cgan_cifar_samples}
\end{figure}

\section{CGAN}
To implement a \gls{CGAN} it is only necessary to modify an existing \gls{GAN} to receive a conditional label. For these experiments, the same base used for the \gls{DCGAN} was used to construct the \gls{CGAN}, the only difference was the addition of another input for the label that will pass through an embedding layer and be incorporated into a channel of the other input, as discussed in \autoref{sub:cgan}.

\subsection{MNIST}
Training the \gls{CGAN} on the \gls{MNIST} dataset produced the results seen in \autoref{fig:cgan_mnist_metrics}.\begin{figure}[hbt]
    \centering
    \caption{Metrics when training a CGAN on MNIST}
    \includegraphics[width=\textwidth]{chapters/Experiments/CGAN/mnist_metrics.pdf}
    \fonte{From the author (2021)}
    \label{fig:cgan_mnist_metrics}
\end{figure}

These results show not only that the metrics have improved, but also that the training is much more stable, for all tests the training produced similar good results.

One of the main benefits of the \gls{CGAN} architecture is the fact that the resulting sample can be controlled by feeding the desired label to the generator. This can be seen in the samples produced by the best model (\texttt{BS=16 SMOOTH=0.9}) shown in \autoref{fig:cgan_mnist_samples}, where each row was conditioned to produce a different set of digits.
\begin{figure}[hbt]
    \centering
    \caption{Samples when training a CGAN on MNIST}
    \includegraphics[width=0.5\textwidth]{chapters/Experiments/CGAN/mnist_samples.png}
    \fonte{From the author (2021)}
    \label{fig:cgan_mnist_samples}
\end{figure}


\subsection{Fashion MNIST}
Training the \gls{CGAN} on the Fashion MNIST dataset produced the results seen in \autoref{fig:cgan_fashion_metrics}.
\begin{figure}[hbt]
    \centering
    \caption{Metrics when training a CGAN on Fashion MNIST}
    \includegraphics[width=\textwidth]{chapters/Experiments/CGAN/fashion_metrics.pdf}
    \fonte{From the author (2021)}
    \label{fig:cgan_fashion_metrics}
\end{figure}

The same behaviour seen for \gls{MNIST} can also be seen here, the overall quality increased, this time however the level of stability is not as strong. The samples produced by the best test (\texttt{BS=32}) are shown in \autoref{fig:cgan_fashion_samples}.
\begin{figure}[hbt]
    \centering
    \caption{Samples when training a CGAN on Fashion MNIST}
    \includegraphics[width=0.5\textwidth]{chapters/Experiments/CGAN/fashion_samples.png}
    \fonte{From the author (2021)}
    \label{fig:cgan_fashion_samples}
\end{figure}


\subsection{CIFAR-10}
Training the \gls{CGAN} on the \gls{CIFAR}-10 dataset produced the results seen in \autoref{fig:cgan_cifar_metrics}. In these experiments it was also tested the \gls{beta_1} term of Adam to see if some momentum in the updates would be beneficial.
\begin{figure}[hbt]
    \centering
    \caption{Metrics when training a CGAN on CIFAR-10}
    \includegraphics[width=\textwidth]{chapters/Experiments/CGAN/cifar_metrics.pdf}
    \fonte{From the author (2021)}
    \label{fig:cgan_cifar_metrics}
\end{figure}

To better understand how each component affects the results it is useful to highlight them separately. \autoref{fig:cgan_cifar_upsampling} shows how the metrics evolve for the different upsampling techniques. The same behaviour observed previously is also seen here, still the transposed convolutions perform better, followed by nearest neighbour and bilinear upsampling.
\begin{figure}[hbt]
    \centering
    \caption{Effects of upsampling when training a CGAN on CIFAR-10}
    \includegraphics[width=\textwidth]{chapters/Experiments/CGAN/cifar_upsampling.pdf}
    \fonte{From the author (2021)}
    \label{fig:cgan_cifar_upsampling}
\end{figure}

In \autoref{fig:cgan_cifar_beta1} it is possible to see the effect of the momentum term \gls{beta_1} in training. The results support the idea that momentum does not help for \gls{CIFAR}-10. Also note how the bad performing cases of transposed convolutions seen in \autoref{fig:cgan_cifar_upsampling} are because the use of momentum.
\begin{figure}[hbt]
    \centering
    \caption{Effects of momentum when training a CGAN on CIFAR-10}
    \includegraphics[width=\textwidth]{chapters/Experiments/CGAN/cifar_beta1.pdf}
    \fonte{From the author (2021)}
    \label{fig:cgan_cifar_beta1}
\end{figure}

Lastly, the effects of label smoothing can be seen in \autoref{fig:cgan_cifar_smoothing}. The impact is still not very significative, although it can be seen a small tendency for better values of the \gls{FCD}.
\begin{figure}[hbt]
    \centering
    \caption{Effects of label smoothing when training a CGAN on CIFAR-10}
    \includegraphics[width=\textwidth]{chapters/Experiments/CGAN/cifar_beta1.pdf}
    \fonte{From the author (2021)}
    \label{fig:cgan_cifar_smoothing}
\end{figure}

The samples produced by the best test (\texttt{$\beta_1$=0.0 UP=TrpConv SMOOTH=0.9}) are shown in \autoref{fig:cgan_cifar_samples}.
\begin{figure}[hbt]
    \centering
    \caption{Samples when training a CGAN on CIFAR-10}
    \includegraphics[width=0.65\textwidth]{chapters/Experiments/CGAN/cifar_samples.pdf}
    \fonte{From the author (2021)}
    \label{fig:cgan_cifar_samples}
\end{figure}

\section{CGAN}
To implement a \gls{CGAN} it is only necessary to modify an existing \gls{GAN} to receive a conditional label. For these experiments, the same base used for the \gls{DCGAN} was used to construct the \gls{CGAN}, the only difference was the addition of another input for the label that will pass through an embedding layer and be incorporated into a channel of the other input, as discussed in \autoref{sub:cgan}.

\subsection{MNIST}
Training the \gls{CGAN} on the \gls{MNIST} dataset produced the results seen in \autoref{fig:cgan_mnist_metrics}.\begin{figure}[hbt]
    \centering
    \caption{Metrics when training a CGAN on MNIST}
    \includegraphics[width=\textwidth]{chapters/Experiments/CGAN/mnist_metrics.pdf}
    \fonte{From the author (2021)}
    \label{fig:cgan_mnist_metrics}
\end{figure}

These results show not only that the metrics have improved, but also that the training is much more stable, for all tests the training produced similar good results.

One of the main benefits of the \gls{CGAN} architecture is the fact that the resulting sample can be controlled by feeding the desired label to the generator. This can be seen in the samples produced by the best model (\texttt{BS=16 SMOOTH=0.9}) shown in \autoref{fig:cgan_mnist_samples}, where each row was conditioned to produce a different set of digits.
\begin{figure}[hbt]
    \centering
    \caption{Samples when training a CGAN on MNIST}
    \includegraphics[width=0.5\textwidth]{chapters/Experiments/CGAN/mnist_samples.png}
    \fonte{From the author (2021)}
    \label{fig:cgan_mnist_samples}
\end{figure}


\subsection{Fashion MNIST}
Training the \gls{CGAN} on the Fashion MNIST dataset produced the results seen in \autoref{fig:cgan_fashion_metrics}.
\begin{figure}[hbt]
    \centering
    \caption{Metrics when training a CGAN on Fashion MNIST}
    \includegraphics[width=\textwidth]{chapters/Experiments/CGAN/fashion_metrics.pdf}
    \fonte{From the author (2021)}
    \label{fig:cgan_fashion_metrics}
\end{figure}

The same behaviour seen for \gls{MNIST} can also be seen here, the overall quality increased, this time however the level of stability is not as strong. The samples produced by the best test (\texttt{BS=32}) are shown in \autoref{fig:cgan_fashion_samples}.
\begin{figure}[hbt]
    \centering
    \caption{Samples when training a CGAN on Fashion MNIST}
    \includegraphics[width=0.5\textwidth]{chapters/Experiments/CGAN/fashion_samples.png}
    \fonte{From the author (2021)}
    \label{fig:cgan_fashion_samples}
\end{figure}


\subsection{CIFAR-10}
Training the \gls{CGAN} on the \gls{CIFAR}-10 dataset produced the results seen in \autoref{fig:cgan_cifar_metrics}. In these experiments it was also tested the \gls{beta_1} term of Adam to see if some momentum in the updates would be beneficial.
\begin{figure}[hbt]
    \centering
    \caption{Metrics when training a CGAN on CIFAR-10}
    \includegraphics[width=\textwidth]{chapters/Experiments/CGAN/cifar_metrics.pdf}
    \fonte{From the author (2021)}
    \label{fig:cgan_cifar_metrics}
\end{figure}

To better understand how each component affects the results it is useful to highlight them separately. \autoref{fig:cgan_cifar_upsampling} shows how the metrics evolve for the different upsampling techniques. The same behaviour observed previously is also seen here, still the transposed convolutions perform better, followed by nearest neighbour and bilinear upsampling.
\begin{figure}[hbt]
    \centering
    \caption{Effects of upsampling when training a CGAN on CIFAR-10}
    \includegraphics[width=\textwidth]{chapters/Experiments/CGAN/cifar_upsampling.pdf}
    \fonte{From the author (2021)}
    \label{fig:cgan_cifar_upsampling}
\end{figure}

In \autoref{fig:cgan_cifar_beta1} it is possible to see the effect of the momentum term \gls{beta_1} in training. The results support the idea that momentum does not help for \gls{CIFAR}-10. Also note how the bad performing cases of transposed convolutions seen in \autoref{fig:cgan_cifar_upsampling} are because the use of momentum.
\begin{figure}[hbt]
    \centering
    \caption{Effects of momentum when training a CGAN on CIFAR-10}
    \includegraphics[width=\textwidth]{chapters/Experiments/CGAN/cifar_beta1.pdf}
    \fonte{From the author (2021)}
    \label{fig:cgan_cifar_beta1}
\end{figure}

Lastly, the effects of label smoothing can be seen in \autoref{fig:cgan_cifar_smoothing}. The impact is still not very significative, although it can be seen a small tendency for better values of the \gls{FCD}.
\begin{figure}[hbt]
    \centering
    \caption{Effects of label smoothing when training a CGAN on CIFAR-10}
    \includegraphics[width=\textwidth]{chapters/Experiments/CGAN/cifar_beta1.pdf}
    \fonte{From the author (2021)}
    \label{fig:cgan_cifar_smoothing}
\end{figure}

The samples produced by the best test (\texttt{$\beta_1$=0.0 UP=TrpConv SMOOTH=0.9}) are shown in \autoref{fig:cgan_cifar_samples}.
\begin{figure}[hbt]
    \centering
    \caption{Samples when training a CGAN on CIFAR-10}
    \includegraphics[width=0.65\textwidth]{chapters/Experiments/CGAN/cifar_samples.pdf}
    \fonte{From the author (2021)}
    \label{fig:cgan_cifar_samples}
\end{figure}

\section{CGAN}
To implement a \gls{CGAN} it is only necessary to modify an existing \gls{GAN} to receive a conditional label. For these experiments, the same base used for the \gls{DCGAN} was used to construct the \gls{CGAN}, the only difference was the addition of another input for the label that will pass through an embedding layer and be incorporated into a channel of the other input, as discussed in \autoref{sub:cgan}.

\subsection{MNIST}
Training the \gls{CGAN} on the \gls{MNIST} dataset produced the results seen in \autoref{fig:cgan_mnist_metrics}.\begin{figure}[hbt]
    \centering
    \caption{Metrics when training a CGAN on MNIST}
    \includegraphics[width=\textwidth]{chapters/Experiments/CGAN/mnist_metrics.pdf}
    \fonte{From the author (2021)}
    \label{fig:cgan_mnist_metrics}
\end{figure}

These results show not only that the metrics have improved, but also that the training is much more stable, for all tests the training produced similar good results.

One of the main benefits of the \gls{CGAN} architecture is the fact that the resulting sample can be controlled by feeding the desired label to the generator. This can be seen in the samples produced by the best model (\texttt{BS=16 SMOOTH=0.9}) shown in \autoref{fig:cgan_mnist_samples}, where each row was conditioned to produce a different set of digits.
\begin{figure}[hbt]
    \centering
    \caption{Samples when training a CGAN on MNIST}
    \includegraphics[width=0.5\textwidth]{chapters/Experiments/CGAN/mnist_samples.png}
    \fonte{From the author (2021)}
    \label{fig:cgan_mnist_samples}
\end{figure}


\subsection{Fashion MNIST}
Training the \gls{CGAN} on the Fashion MNIST dataset produced the results seen in \autoref{fig:cgan_fashion_metrics}.
\begin{figure}[hbt]
    \centering
    \caption{Metrics when training a CGAN on Fashion MNIST}
    \includegraphics[width=\textwidth]{chapters/Experiments/CGAN/fashion_metrics.pdf}
    \fonte{From the author (2021)}
    \label{fig:cgan_fashion_metrics}
\end{figure}

The same behaviour seen for \gls{MNIST} can also be seen here, the overall quality increased, this time however the level of stability is not as strong. The samples produced by the best test (\texttt{BS=32}) are shown in \autoref{fig:cgan_fashion_samples}.
\begin{figure}[hbt]
    \centering
    \caption{Samples when training a CGAN on Fashion MNIST}
    \includegraphics[width=0.5\textwidth]{chapters/Experiments/CGAN/fashion_samples.png}
    \fonte{From the author (2021)}
    \label{fig:cgan_fashion_samples}
\end{figure}


\subsection{CIFAR-10}
Training the \gls{CGAN} on the \gls{CIFAR}-10 dataset produced the results seen in \autoref{fig:cgan_cifar_metrics}. In these experiments it was also tested the \gls{beta_1} term of Adam to see if some momentum in the updates would be beneficial.
\begin{figure}[hbt]
    \centering
    \caption{Metrics when training a CGAN on CIFAR-10}
    \includegraphics[width=\textwidth]{chapters/Experiments/CGAN/cifar_metrics.pdf}
    \fonte{From the author (2021)}
    \label{fig:cgan_cifar_metrics}
\end{figure}

To better understand how each component affects the results it is useful to highlight them separately. \autoref{fig:cgan_cifar_upsampling} shows how the metrics evolve for the different upsampling techniques. The same behaviour observed previously is also seen here, still the transposed convolutions perform better, followed by nearest neighbour and bilinear upsampling.
\begin{figure}[hbt]
    \centering
    \caption{Effects of upsampling when training a CGAN on CIFAR-10}
    \includegraphics[width=\textwidth]{chapters/Experiments/CGAN/cifar_upsampling.pdf}
    \fonte{From the author (2021)}
    \label{fig:cgan_cifar_upsampling}
\end{figure}

In \autoref{fig:cgan_cifar_beta1} it is possible to see the effect of the momentum term \gls{beta_1} in training. The results support the idea that momentum does not help for \gls{CIFAR}-10. Also note how the bad performing cases of transposed convolutions seen in \autoref{fig:cgan_cifar_upsampling} are because the use of momentum.
\begin{figure}[hbt]
    \centering
    \caption{Effects of momentum when training a CGAN on CIFAR-10}
    \includegraphics[width=\textwidth]{chapters/Experiments/CGAN/cifar_beta1.pdf}
    \fonte{From the author (2021)}
    \label{fig:cgan_cifar_beta1}
\end{figure}

Lastly, the effects of label smoothing can be seen in \autoref{fig:cgan_cifar_smoothing}. The impact is still not very significative, although it can be seen a small tendency for better values of the \gls{FCD}.
\begin{figure}[hbt]
    \centering
    \caption{Effects of label smoothing when training a CGAN on CIFAR-10}
    \includegraphics[width=\textwidth]{chapters/Experiments/CGAN/cifar_beta1.pdf}
    \fonte{From the author (2021)}
    \label{fig:cgan_cifar_smoothing}
\end{figure}

The samples produced by the best test (\texttt{$\beta_1$=0.0 UP=TrpConv SMOOTH=0.9}) are shown in \autoref{fig:cgan_cifar_samples}.
\begin{figure}[hbt]
    \centering
    \caption{Samples when training a CGAN on CIFAR-10}
    \includegraphics[width=0.65\textwidth]{chapters/Experiments/CGAN/cifar_samples.pdf}
    \fonte{From the author (2021)}
    \label{fig:cgan_cifar_samples}
\end{figure}

\section{CGAN}
To implement a \gls{CGAN} it is only necessary to modify an existing \gls{GAN} to receive a conditional label. For these experiments, the same base used for the \gls{DCGAN} was used to construct the \gls{CGAN}, the only difference was the addition of another input for the label that will pass through an embedding layer and be incorporated into a channel of the other input, as discussed in \autoref{sub:cgan}.

\subsection{MNIST}
Training the \gls{CGAN} on the \gls{MNIST} dataset produced the results seen in \autoref{fig:cgan_mnist_metrics}.\begin{figure}[hbt]
    \centering
    \caption{Metrics when training a CGAN on MNIST}
    \includegraphics[width=\textwidth]{chapters/Experiments/CGAN/mnist_metrics.pdf}
    \fonte{From the author (2021)}
    \label{fig:cgan_mnist_metrics}
\end{figure}

These results show not only that the metrics have improved, but also that the training is much more stable, for all tests the training produced similar good results.

One of the main benefits of the \gls{CGAN} architecture is the fact that the resulting sample can be controlled by feeding the desired label to the generator. This can be seen in the samples produced by the best model (\texttt{BS=16 SMOOTH=0.9}) shown in \autoref{fig:cgan_mnist_samples}, where each row was conditioned to produce a different set of digits.
\begin{figure}[hbt]
    \centering
    \caption{Samples when training a CGAN on MNIST}
    \includegraphics[width=0.5\textwidth]{chapters/Experiments/CGAN/mnist_samples.png}
    \fonte{From the author (2021)}
    \label{fig:cgan_mnist_samples}
\end{figure}


\subsection{Fashion MNIST}
Training the \gls{CGAN} on the Fashion MNIST dataset produced the results seen in \autoref{fig:cgan_fashion_metrics}.
\begin{figure}[hbt]
    \centering
    \caption{Metrics when training a CGAN on Fashion MNIST}
    \includegraphics[width=\textwidth]{chapters/Experiments/CGAN/fashion_metrics.pdf}
    \fonte{From the author (2021)}
    \label{fig:cgan_fashion_metrics}
\end{figure}

The same behaviour seen for \gls{MNIST} can also be seen here, the overall quality increased, this time however the level of stability is not as strong. The samples produced by the best test (\texttt{BS=32}) are shown in \autoref{fig:cgan_fashion_samples}.
\begin{figure}[hbt]
    \centering
    \caption{Samples when training a CGAN on Fashion MNIST}
    \includegraphics[width=0.5\textwidth]{chapters/Experiments/CGAN/fashion_samples.png}
    \fonte{From the author (2021)}
    \label{fig:cgan_fashion_samples}
\end{figure}


\subsection{CIFAR-10}
Training the \gls{CGAN} on the \gls{CIFAR}-10 dataset produced the results seen in \autoref{fig:cgan_cifar_metrics}. In these experiments it was also tested the \gls{beta_1} term of Adam to see if some momentum in the updates would be beneficial.
\begin{figure}[hbt]
    \centering
    \caption{Metrics when training a CGAN on CIFAR-10}
    \includegraphics[width=\textwidth]{chapters/Experiments/CGAN/cifar_metrics.pdf}
    \fonte{From the author (2021)}
    \label{fig:cgan_cifar_metrics}
\end{figure}

To better understand how each component affects the results it is useful to highlight them separately. \autoref{fig:cgan_cifar_upsampling} shows how the metrics evolve for the different upsampling techniques. The same behaviour observed previously is also seen here, still the transposed convolutions perform better, followed by nearest neighbour and bilinear upsampling.
\begin{figure}[hbt]
    \centering
    \caption{Effects of upsampling when training a CGAN on CIFAR-10}
    \includegraphics[width=\textwidth]{chapters/Experiments/CGAN/cifar_upsampling.pdf}
    \fonte{From the author (2021)}
    \label{fig:cgan_cifar_upsampling}
\end{figure}

In \autoref{fig:cgan_cifar_beta1} it is possible to see the effect of the momentum term \gls{beta_1} in training. The results support the idea that momentum does not help for \gls{CIFAR}-10. Also note how the bad performing cases of transposed convolutions seen in \autoref{fig:cgan_cifar_upsampling} are because the use of momentum.
\begin{figure}[hbt]
    \centering
    \caption{Effects of momentum when training a CGAN on CIFAR-10}
    \includegraphics[width=\textwidth]{chapters/Experiments/CGAN/cifar_beta1.pdf}
    \fonte{From the author (2021)}
    \label{fig:cgan_cifar_beta1}
\end{figure}

Lastly, the effects of label smoothing can be seen in \autoref{fig:cgan_cifar_smoothing}. The impact is still not very significative, although it can be seen a small tendency for better values of the \gls{FCD}.
\begin{figure}[hbt]
    \centering
    \caption{Effects of label smoothing when training a CGAN on CIFAR-10}
    \includegraphics[width=\textwidth]{chapters/Experiments/CGAN/cifar_beta1.pdf}
    \fonte{From the author (2021)}
    \label{fig:cgan_cifar_smoothing}
\end{figure}

The samples produced by the best test (\texttt{$\beta_1$=0.0 UP=TrpConv SMOOTH=0.9}) are shown in \autoref{fig:cgan_cifar_samples}.
\begin{figure}[hbt]
    \centering
    \caption{Samples when training a CGAN on CIFAR-10}
    \includegraphics[width=0.65\textwidth]{chapters/Experiments/CGAN/cifar_samples.pdf}
    \fonte{From the author (2021)}
    \label{fig:cgan_cifar_samples}
\end{figure}

\section{CGAN}
To implement a \gls{CGAN} it is only necessary to modify an existing \gls{GAN} to receive a conditional label. For these experiments, the same base used for the \gls{DCGAN} was used to construct the \gls{CGAN}, the only difference was the addition of another input for the label that will pass through an embedding layer and be incorporated into a channel of the other input, as discussed in \autoref{sub:cgan}.

\subsection{MNIST}
Training the \gls{CGAN} on the \gls{MNIST} dataset produced the results seen in \autoref{fig:cgan_mnist_metrics}.\begin{figure}[hbt]
    \centering
    \caption{Metrics when training a CGAN on MNIST}
    \includegraphics[width=\textwidth]{chapters/Experiments/CGAN/mnist_metrics.pdf}
    \fonte{From the author (2021)}
    \label{fig:cgan_mnist_metrics}
\end{figure}

These results show not only that the metrics have improved, but also that the training is much more stable, for all tests the training produced similar good results.

One of the main benefits of the \gls{CGAN} architecture is the fact that the resulting sample can be controlled by feeding the desired label to the generator. This can be seen in the samples produced by the best model (\texttt{BS=16 SMOOTH=0.9}) shown in \autoref{fig:cgan_mnist_samples}, where each row was conditioned to produce a different set of digits.
\begin{figure}[hbt]
    \centering
    \caption{Samples when training a CGAN on MNIST}
    \includegraphics[width=0.5\textwidth]{chapters/Experiments/CGAN/mnist_samples.png}
    \fonte{From the author (2021)}
    \label{fig:cgan_mnist_samples}
\end{figure}


\subsection{Fashion MNIST}
Training the \gls{CGAN} on the Fashion MNIST dataset produced the results seen in \autoref{fig:cgan_fashion_metrics}.
\begin{figure}[hbt]
    \centering
    \caption{Metrics when training a CGAN on Fashion MNIST}
    \includegraphics[width=\textwidth]{chapters/Experiments/CGAN/fashion_metrics.pdf}
    \fonte{From the author (2021)}
    \label{fig:cgan_fashion_metrics}
\end{figure}

The same behaviour seen for \gls{MNIST} can also be seen here, the overall quality increased, this time however the level of stability is not as strong. The samples produced by the best test (\texttt{BS=32}) are shown in \autoref{fig:cgan_fashion_samples}.
\begin{figure}[hbt]
    \centering
    \caption{Samples when training a CGAN on Fashion MNIST}
    \includegraphics[width=0.5\textwidth]{chapters/Experiments/CGAN/fashion_samples.png}
    \fonte{From the author (2021)}
    \label{fig:cgan_fashion_samples}
\end{figure}


\subsection{CIFAR-10}
Training the \gls{CGAN} on the \gls{CIFAR}-10 dataset produced the results seen in \autoref{fig:cgan_cifar_metrics}. In these experiments it was also tested the \gls{beta_1} term of Adam to see if some momentum in the updates would be beneficial.
\begin{figure}[hbt]
    \centering
    \caption{Metrics when training a CGAN on CIFAR-10}
    \includegraphics[width=\textwidth]{chapters/Experiments/CGAN/cifar_metrics.pdf}
    \fonte{From the author (2021)}
    \label{fig:cgan_cifar_metrics}
\end{figure}

To better understand how each component affects the results it is useful to highlight them separately. \autoref{fig:cgan_cifar_upsampling} shows how the metrics evolve for the different upsampling techniques. The same behaviour observed previously is also seen here, still the transposed convolutions perform better, followed by nearest neighbour and bilinear upsampling.
\begin{figure}[hbt]
    \centering
    \caption{Effects of upsampling when training a CGAN on CIFAR-10}
    \includegraphics[width=\textwidth]{chapters/Experiments/CGAN/cifar_upsampling.pdf}
    \fonte{From the author (2021)}
    \label{fig:cgan_cifar_upsampling}
\end{figure}

In \autoref{fig:cgan_cifar_beta1} it is possible to see the effect of the momentum term \gls{beta_1} in training. The results support the idea that momentum does not help for \gls{CIFAR}-10. Also note how the bad performing cases of transposed convolutions seen in \autoref{fig:cgan_cifar_upsampling} are because the use of momentum.
\begin{figure}[hbt]
    \centering
    \caption{Effects of momentum when training a CGAN on CIFAR-10}
    \includegraphics[width=\textwidth]{chapters/Experiments/CGAN/cifar_beta1.pdf}
    \fonte{From the author (2021)}
    \label{fig:cgan_cifar_beta1}
\end{figure}

Lastly, the effects of label smoothing can be seen in \autoref{fig:cgan_cifar_smoothing}. The impact is still not very significative, although it can be seen a small tendency for better values of the \gls{FCD}.
\begin{figure}[hbt]
    \centering
    \caption{Effects of label smoothing when training a CGAN on CIFAR-10}
    \includegraphics[width=\textwidth]{chapters/Experiments/CGAN/cifar_beta1.pdf}
    \fonte{From the author (2021)}
    \label{fig:cgan_cifar_smoothing}
\end{figure}

The samples produced by the best test (\texttt{$\beta_1$=0.0 UP=TrpConv SMOOTH=0.9}) are shown in \autoref{fig:cgan_cifar_samples}.
\begin{figure}[hbt]
    \centering
    \caption{Samples when training a CGAN on CIFAR-10}
    \includegraphics[width=0.65\textwidth]{chapters/Experiments/CGAN/cifar_samples.pdf}
    \fonte{From the author (2021)}
    \label{fig:cgan_cifar_samples}
\end{figure}

% ----------------------------------------------------------


% -- ELEMENTOS PÓS-TEXTUAIS  -------------------------------
\postextual

% Referências
\begingroup
    \SingleSpacing\printbibliography[title=REFERÊNCIAS]
\endgroup

%Apêndices
\begin{apendicesenv}
	\chapter{Specifications of the Machine} \label{apd:machine_specs}
The main components of the machine which ran the experiments consist of:
\begin{itemize}
    \item \textbf{Processor} AMD FX(tm)-6100 Six-Core 3.3 GHz
    \item \textbf{RAM} 4.0 GB
    \item \textbf{Video Card} NVIDIA GeForce GTX 1050, 2.0GB Dedicated Memory, 4.0GB Total Memory
    \item \textbf{Tensorflow} version 2.3.0
\end{itemize}
\end{apendicesenv}

% Anexos
% \begin{anexosenv}
% 	\input{aftertext/anexo_a}
% \end{anexosenv}

% ----------------------------------------------------------


\end{document}
